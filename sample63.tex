
%% Beginning of file 'sample63.tex'
%%
%% Modified 2019 June
%%
%% This is a sample manuscript marked up using the
%% AASTeX v6.3 LaTeX 2e macros.
%%
%% AASTeX is now based on Alexey Vikhlinin's emulateapj.cls 
%% (Copyright 2000-2015).  See the classfile for details.

%% AASTeX requires revtex4-1.cls (http://publish.aps.org/revtex4/) and
%% other external packages (latexsym, graphicx, amssymb, longtable, and epsf).
%% All of these external packages should already be present in the modern TeX 
%% distributions.  If not they can also be obtained at www.ctan.org.

%% The first piece of markup in an AASTeX v6.x document is the \documentclass
%% command. LaTeX will ignore any data that comes before this command. The 
%% documentclass can take an optional argument to modify the output style.
%% The command below calls the preprint style which will produce a tightly 
%% typeset, one-column, single-spaced document.  It is the default and thus
%% does not need to be explicitly stated.
%%
%%
%% using aastex version 6.3
\documentclass{aastex63}


\newcommand{\vdag}{(v)^\dagger}
\newcommand\aastex{AAS\TeX}
\newcommand\latex{La\TeX}

%% Reintroduced the \received and \accepted commands from AASTeX v5.2
%\received{June 1, 2019}
%\revised{January 10, 2019}
%\accepted{\today}
%% Command to document which AAS Journal the manuscript was submitted to.
%% Adds "Submitted to " the argument.
\submitjournal{AJ}

%% For manuscript that include authors in collaborations, AASTeX v6.3
%% builds on the \collaboration command to allow greater freedom to 
%% keep the traditional author+affiliation information but only show
%% subsets. The \collaboration command now must appear AFTER the group
%% of authors in the collaboration and it takes TWO arguments. The last
%% is still the collaboration identifier. The text given in this
%% argument is what will be shown in the manuscript. The first argument
%% is the number of author above the \collaboration command to show with
%% the collaboration text. If there are authors that are not part of any
%% collaboration the \nocollaboration command is used. This command takes
%% one argument which is also the number of authors above to show. A
%% dashed line is shown to indicate no collaboration. This example manuscript
%% shows how these commands work to display specific set of authors 
%% on the front page.
%%
%% For manuscript without any need to use \collaboration the 
%% \AuthorCollaborationLimit command from v6.2 can still be used to 
%% show a subset of authors.
%
%\AuthorCollaborationLimit=2
%
%% will only show Schwarz & Muench on the front page of the manuscript
%% (assuming the \collaboration and \nocollaboration commands are
%% commented out).
%%
%% Note that all of the author will be shown in the published article.
%% This feature is meant to be used prior to acceptance to make the
%% front end of a long author article more manageable. Please do not use
%% this functionality for manuscripts with less than 20 authors. Conversely,
%% please do use this when the number of authors exceeds 40.
%%
%% Use \allauthors at the manuscript end to show the full author list.
%% This command should only be used with \AuthorCollaborationLimit is used.

%% The following command can be used to set the latex table counters.  It
%% is needed in this document because it uses a mix of latex tabular and
%% AASTeX deluxetables.  In general it should not be needed.
%\setcounter{table}{1}

%%%%%%%%%%%%%%%%%%%%%%%%%%%%%%%%%%%%%%%%%%%%%%%%%%%%%%%%%%%%%%%%%%%%%%%%%%%%%%%%
%%
%% The following section outlines numerous optional output that
%% can be displayed in the front matter or as running meta-data.
%%
%% If you wish, you may supply running head information, although
%% this information may be modified by the editorial offices.
%\shorttitle{Sample article}
%\shortauthors{Schwarz et al.}
%%
%% You can add a light gray and diagonal water-mark to the first page 
%% with this command:
%% \watermark{text}
%% where "text", e.g. DRAFT, is the text to appear.  If the text is 
%% long you can control the water-mark size with:
%% \setwatermarkfontsize{dimension}
%% where dimension is any recognized LaTeX dimension, e.g. pt, in, etc.
%%
%%%%%%%%%%%%%%%%%%%%%%%%%%%%%%%%%%%%%%%%%%%%%%%%%%%%%%%%%%%%%%%%%%%%%%%%%%%%%%%%
\graphicspath{{./}{figures/}}
%% This is the end of the preamble.  Indicate the beginning of the
%% manuscript itself with \begin{document}.

\begin{document}

\title{Magnetic Reconnection Simulations using WarpX}



%\correspondingauthor{Cameron Yang}
%\email{cameronyang@lbl.gov}

\author{Astro-plasma group}
\affiliation{Lawrence Berkeley National Lab}






%% Note that the \and command from previous versions of AASTeX is now
%% depreciated in this version as it is no longer necessary. AASTeX 
%% automatically takes care of all commas and "and"s between authors names.

%% AASTeX 6.3 has the new \collaboration and \nocollaboration commands to
%% provide the collaboration status of a group of authors. These commands 
%% can be used either before or after the list of corresponding authors. The
%% argument for \collaboration is the collaboration identifier. Authors are
%% encouraged to surround collaboration identifiers with ()s. The 
%% \nocollaboration command takes no argument and exists to indicate that
%% the nearby authors are not part of surrounding collaborations.

%% Mark off the abstract in the ``abstract'' environment. 
\begin{abstract}


\end{abstract}

%% Keywords should appear after the \end{abstract} command. 
%% See the online documentation for the full list of available subject
%% keywords and the rules for their use.
\keywords{editorials, notices --- 
miscellaneous --- catalogs --- surveys}

%% From the front matter, we move on to the body of the paper.
%% Sections are demarcated by \section and \subsection, respectively.
%% Observe the use of the LaTeX \label
%% command after the \subsection to give a symbolic KEY to the
%% subsection for cross-referencing in a \ref command.
%% You can use LaTeX's \ref and \label commands to keep track of
%% cross-references to sections, equations, tables, and figures.
%% That way, if you change the order of any elements, LaTeX will
%% automatically renumber them.
%%
%% We recommend that authors also use the natbib \citep
%% and \citet commands to identify citations.  The citations are
%% tied to the reference list via symbolic KEYs. The KEY corresponds
%% to the KEY in the \bibitem in the reference list below. 

\section{Introduction} 
\label{sec:intro}

\subsection{Literature Review}
The Harris sheet set-up is similar to that described by Werner et al.,\citep{werner2017non}.


\section{Numerical SetUp}



\section{Verification and convergence studies}

\section{Effect of perturbation on rate of magnetic reconnection}

\section{Effect of current sheet thickness of rate of magnetic reconnection}

\section{Conclusions}



\acknowledgments

We thank ..


\bibliography{sample63}{}
\bibliographystyle{aasjournal}

%% This command is needed to show the entire author+affiliation list when
%% the collaboration and author truncation commands are used.  It has to
%% go at the end of the manuscript.
%\allauthors

%% Include this line if you are using the \added, \replaced, \deleted
%% commands to see a summary list of all changes at the end of the article.
%\listofchanges

\end{document}

% End of file `sample63.tex'.
